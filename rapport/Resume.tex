\begin{abstract}
	Ce rapport fait état du projet de 5e année d’école d’ingénieurs à l’INSA Centre Val de Loire. 
	Il a pour but d’exposer un travail visant à coupler l’apprentissage profond avec des procédés de calcul et d’extraction de descripteurs, afin de procéder au suivi d’un objet d’intérêt en temps réel à partir d’un support vidéo.  
\end{abstract}

\section*{Remerciements}
\phantomsection
	En prélude à ce rapport nous tenions à remercier M. Adel Hafiane, Professeur et Responsable du Laboratoire de Vision par Ordinateurs à l’INSA Centre Val de Loire (18000 Bourges, France) et Dr à l’Université du Missouri (Columbia 65211, USA). Nous saluons vivement votre implication tout au long de ce projet tant par vos suggestions que par votre éclairage toujours pertinent dans ce domaine du traitement de l’image. Enfin nous sommes nous deux conscients de l’importance de votre appui dans le cadre de nos recherches de stages respectives. Nous vous ré-adressons à cet effet toute notre gratitude.
	
	Ce projet venant parachever la part in situ de notre cinquième année à l’INSA CVL, nous souhaitons exprimer nos sincères remerciements envers MM. Patrice Clémente et Pascal Berthomé, tous deux ayant respectivement la charge de l’option Architecture et Sécurité du Logiciel et du cycle Sécurité et Technologies Informatique. 
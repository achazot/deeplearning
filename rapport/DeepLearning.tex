\documentclass[a4paper,10pt]{report}
\usepackage[utf8]{inputenc}
\usepackage{hyperref}
\usepackage{svg}


%opening
\title{}
\author{}

\begin{document}

\maketitle

\chapter{Deep Learning}


\section{Machine Learning}

\subsection{Principes généraux}

Le Machine Learning, également connu sous le terme d'apprentissage automatique, est un sous-domaine de l'intelligence artificielle, ayant vu le jour dans
les années 1950\footnote{
A Short History of Machine Learning :
  \hyperref[Mach_Learn_history]{''http://www.forbes.com/sites/bernardmarr/2016/02/19/a-short-history-of-machine-learning-every-manager-should-read/fdc5dfd323ff''}
   \\
   Arthur Samuel: Pioneer in Machine Learning :
   \hyperref[item]{''http://infolab.stanford.edu/pub/voy/museum/samuel.html''}}.
Son champ d'application est aujourd'hui très vaste, et sa principale limite réside en la quantité d'informations exploitables, disponible au sein d'un domaine donné.
L'accroissement de la collecte, du stockage et de la mise à disposition des données que nous connaissons depuis quelques années a permit l'essor de ces algorithmes et leur transposition à de nombreux problèmes :
\begin{itemize}
\item la classification de contenu audio-visuel au sens large, allant de l'image au sujet d'un texte ou d'une revue
\item le filtrage de contenus, tels que les spams ou les intrusions sur les systèmes d'informations
\item le tri et la sélection d'informations les plus pertinentes à délivrer via la publicité ou les flux de contenus des médias sociaux
\item l'analyse de sentiments
\end{itemize}

Plus précisément, ce concept recouvre les systèmes constitués de paramètres réglables, typiquement vus sous la forme de
valeurs vectorielles, en vue de fournir la sortie attendue pour une série de valeurs données en entrée. En outre, ce type d'apprentissage se distingue
par sa capacité à ajuster ses paramètres de manière autonome, en se basant sur l'expérience des données précedemment traîtées.

Dans ce qui va suivre, cette technique sera abordée à la lumière des problèmes de reconnaissance de formes, les entrées dont il sera alors question étant des images ou des vidéos.

\subsection{Architecture des systèmes de reconnaissance de forme}

L'architecture d'un système de reconnaissance de formes comprend deux éléments, à savoir un algorithme dédié à l'extraction des caractéristiques de l'entrée, ainsi qu'un algorithme de classification produisant le résultat.
Dans les années 50, les premiers modèles de reconnaissance étaient constitués d'extracteurs de caractéristiques "faits-main", peu modulaires et très longs à implémenter.
L'algorithme de classification pouvait quant à lui ajuster des paramètres internes, en vue d'améliorer la sortie produite en fonction des résultats précédents. On parle alors d'entraînement.

L'approche par apprentissage profond a permit d'étendre la capacité d'entraînement à l'ensemble de la chaîne de reconnaissance.

\begin{figure}
    \centering
    \makebox[\textwidth]{\includesvg[width=.6\paperwidth]{c1p1s1_schema}}
    \caption{Différentes approches de la reconnaissance de forme}
    \label{fig:Fig1}
\end{figure}

Notre étude se base sur ce dernier modèle, où l'extracteur de caractéristiques, aussi appelé noyau, peut être entraîné.

deep learning : connu depuis 1980 mais répendu depuis 2012
perceptron
capacité de généralisation : faculté d'un système à produire des résultats corrects sur des entrées inconnues, après un phase d'entraînement



\section{Réseaux de neurones}


\end{document}
